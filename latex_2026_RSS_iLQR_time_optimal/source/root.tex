\documentclass[conference]{IEEEtran}
\usepackage{times}
\usepackage[numbers]{natbib}
\usepackage{multicol}
\usepackage[bookmarks=true]{hyperref}

\newcommand{\T}{\mathsf{T}}
\IEEEoverridecommandlockouts

\usepackage{amsmath,amsfonts}
\let\proof\relax
\let\endproof\relax
\usepackage{amsthm}
\usepackage{amssymb}

\usepackage[ruled,vlined]{algorithm2e}
\usepackage{array}
\usepackage[caption=false,font=normalsize,labelfont=sf,textfont=sf]{subfig}
\usepackage{textcomp}
\usepackage{stfloats}
\usepackage{url}
\usepackage{verbatim}
\usepackage{graphicx}
\hyphenation{op-tical net-works semi-conduc-tor IEEE-Xplore}

\usepackage{tikz}
\usetikzlibrary{shapes.geometric, arrows, positioning, calc, shapes.symbols}
\usepackage{booktabs}
\usepackage{multirow}
\usepackage{lscape}
\usepackage{tabularx}
\usepackage{comment}
\usepackage{color}
\usepackage{algorithm}
\usepackage[noend]{algpseudocode}
\usepackage{xspace}

\newtheorem{assumption}{Assumption}
\newtheorem{theorem}{Theorem}
\newtheorem{lemma}{Lemma}
\newtheorem{definition}{Definition}
\newtheorem{corollary}{Corollary}
\newtheorem{problem}{Problem}
\newtheorem{remark}{Remark}

\newcommand{\red}{\color{red}}
\newcommand{\blue}{\color{blue}}
\newcommand{\green}{\color{green}}

\newcommand\abbrAl{HOP\xspace}
\newcommand\abbrAlg{HOP-LQR\xspace}
\newcommand\abbrAlgg{HOP-DDP\xspace}

\pdfinfo{
   /Author (Author Names Omitted)
   /Title  (HOP: Fast Differential Dynamic Programming for Horizon-Optimal Trajectory Planning)
   /Subject (Robotics)
   /Keywords (Motion Planning; Optimal Control)
}

\begin{document}

\title{HOP: Fast Differential Dynamic Programming for Horizon-Optimal Trajectory Planning}

% \author{Author Names Omitted for Anonymous Review. Paper-ID 495}

\author{Miaomiao Dai$^{1}$, Zhongqiang Ren$^{1\dagger}$%
\thanks{$^{1}$Global College, Shanghai Jiao Tong University.}%
\thanks{$\dagger$ Corresponding author. {\tt\footnotesize zhongqiang.ren@sjtu.edu.cn}.}
}

\maketitle

\begin{abstract}
    This paper considers a Horizon-Optimal Control problem that seeks a dynamically feasible trajectory while minimizing the planning horizon, which is a fundamental problem in robotics with numerous applications.
While many famous optimal control methods, such as LQR, iLQR/DDP, are well studied and deployed on various robots, they often have a fixed planning horizon, and their horizon-optimal counterparts are still undiscovered.
The best result in the literature solves the horizon-optimal LQR problem by shifting the horizon and reusing the value functions computed by the Riccati recursion, which leads to an efficient algorithm.
However, this approach is limited to LQR with time-invariant dynamics and costs only.
This paper finds that the Riccati recursion can be reformulated into a form of Linear Fractional Transformation (LFT), which enjoys the structure that enables efficient computational reuse even for non-stationary dynamics and costs.
Based on this insight, we develop a new efficient algorithm to solve Horizon-Optimal Time-Varying LQR problem to optimality, and further fuse it with DDP to handle general non-quadratic costs and nonlinear dynamics.
Results show, our approach always finds the same optimal solution as a naive brute force baseline method, while running up to 40 times faster.
For nonlinear dynamics, our method always finds better solutions than approximation using time-invariant LQR.


% This paper considers a Horizon-Optimal Control problem that seeks a dynamically feasible trajectory while minimizing the planning horizon, which is a fundamental problem in robotics with numerous applications.
% While many famous optimal control methods, such as LQR, iLQR/DDP, are well studied and deployed on various robots, they often have a fixed planning horizon, and their horizon-optimal counterparts are still undiscovered.
% The best result in the literature solves the horizon-optimal LQR problem by shifting the horizon and reusing the value functions computed by the Riccati recursion, which leads to an efficient algorithm.
% However, this approach is limited to LQR with time-invariant dynamics and costs.
% This paper finds that the Riccati recursion can be reformulated into a form of Linear Fractional Transformation (LFT), which enjoys the structure that enables efficient computation by reusing a new form of value functions during the backwards pass, even for non-stationary dynamics and costs.
% Based on this insight, we develop a new efficient algorithm to solve Horizon-Optimal Time-Varying LQR problem to optimality, and further fuse it with DDP to handle general non-quadratic costs and nonlinear dynamics.
% Results show, our approach always finds the same optimal solution as a naive brute force baseline method, while running up to 40 times faster.
% For nonlinear dynamics, our method always finds better solutions than approximation using time-invariant LQR.
\end{abstract}

% \begin{IEEEkeywords}
% Motion and path planning, multi-robot systems, path planning for multiple mobile robots or agents.
% \end{IEEEkeywords}

\graphicspath{{./figures/}}

\section{Introduction}\label{milp:sec:intro}

Optimal Control (OC) problems seek a dynamically feasible trajectory that minimizes a cost functional defined along the trajectory, which is of fundamental importance in robotics, and is the basis for numerous research topics such as agile flight of drones~\cite{wang2022geometrically,foehn2021time}, dynamic balancing of legged robots~\cite{kuindersma2016atlas}, trajectory planning for multiple robots~\cite{2025_RAL_CP_MILP} or information search~\cite{dong2025time}.
Many famous OC approaches, such as Linear Quadratic Regulator (LQR)~\cite{andersonmoore1990lqr}, iterative LQR (iLQR)~\cite{li2004ilqr}, and differential dynamic programming (DDP)~\cite{jacobson1970ddp}, often assume a fixed planning horizon, which limits their usage especially in applications such as drone racing~\cite{wang2022geometrically,foehn2021time}, where the horizon itself is part of the objectives to be minimized.

To bypass fixed planning horizons, horizon-optimal (or time-optimal) OC was studied and existing research can be roughly classified into two categories.
The first class of methods includes the horizon itself as a decision variable and formulates the problem as a nonlinear program~\cite{wang2022geometrically,foehn2021time, betts1998survey,rao2009survey}.
While being general to handle a variety of systems and tasks, this class of methods often suffers from local minima and can be computationally expensive.
The second class of methods discretizes the horizon into time steps, and seeks to extend the classic OC methods, such as LQR and DDP, to find an optimal number of time steps while optimizing the trajectories~\cite{stachowicz2021ohmp,verriest1991lqmt,elalami1998dlqmt,de2019free}.
While enjoying theoretical properties (such as solution optimality guarantees like LQR) and being computationally efficient, this class of methods is currently limited to only a few special cases that limits their usage.
\begin{figure}[t]
    \centering
    \includegraphics[width=\linewidth]{header.png}
    \vspace{-6mm}
    \caption{For a quadrotor dynamics with 12 degrees of freedom, our \abbrAlgg finds the best solution trajectory with horizon $T^*=26$ in $2.9$ seconds while the DDP based on time-invariant LQR ~\cite{stachowicz2021ohmp} and NLP converges to local minima $T^*=48,30$ in $3.8$ and $71.8$ seconds respectively. 
    % approximates using 
    % We compare \abbrAl (Ours) with two baselines: 
    % NLP, which treats the horizon as a variable in a nonlinear program; and
    % OP, the {one-pass/shift-horizon} baseline that reuses a single backward pass ~\cite{stachowicz2021ohmp}.
    % In this example, \abbrAl get the minimal cost while being faster, whereas OP selects a longer horizon and gets stuck in a worse local minimum; NLP is much slower.
    }
    \vspace{-4mm}
    \label{fig:header}
\end{figure}

This paper is interested in the second class of methods, and develops new fast algorithms for horizon-optimal OC.
Of close relevance to this paper, prior work~\cite{stachowicz2021ohmp} observes that: when solving LQR problem with Riccati recursion backwards from the end of the horizon to the starting time, as long as the dynamics and costs are stationary (i.e., time-invariant), ``shifting the horizon'' does not affect the value functions which allows reusing the value functions to efficiently solve the horizon-optimal LQR problem.
Based on this result, prior work~\cite{stachowicz2021ohmp} further develops iLQR/DDP-like algorithms for nonlinear cases.
% However, this idea of shifting the horizon is limited to stationary dynamics and costs.
% which, in other words, can only handle time-invariant LQR problems.
However, for non-stationary dynamics or costs, the value functions are time-varying and thus cannot be reused any more, and this idea of shifting the horizon fails.
This also limits the resulting iLQR/DDP, since linearizing the nonlinear dynamics along a trajectory at different times or positions can naturally lead to time-varying dynamics.
% To make matters worse, the idea of shifting the horizon cannot be generalized to time-varying systems.

To address this challenge and bypass the assumption on stationary dynamics and costs, this paper develops a new approach \abbrAl (Horizon-Optimal Planning). 
The key idea in \abbrAl is based on an observation that, the Riccati recursion can be reformulated into a form of Linear Fractional Transformation (LFT), which enjoys the structure that enables efficient computation by reusing a new form of value functions during the backwards pass, even for non-stationary dynamics and costs.
% This structure decouples the backward recursion from the specific terminal time $N$, allowing us to query the optimal cost for any candidate horizon using the same pre-computed propagator sequence. 
Based on this idea, we develop \abbrAlg that can solve Horizon-Optimal Time-Varying LQR problem to optimality, and we show that its runtime complexity is same as a regular Riccati recursion for the basic LQR problem.
Based on \abbrAlg, we further develop \abbrAlgg by introducing an augmented state space formulation, which allows solving horizon-optimal OC problems with general nonlinear dynamics and non-quadratic costs efficiently.
% To handle non-stationary dynamics and costs, a naive approach is to run $N$ Riccati recursion (with $N$ being the largest possible horizon), one for each possible horizon, which leads to a $O(N^2n^3)$ complexity 

We compare our \abbrAl against several baselines on different dynamic systems.
Experimental results show that, our \abbrAl always finds the same optimal solution as a naive brute force baseline method, while running up to 40 times faster, for both linear and nonlinear systems.
In comparison to the shift horizon baseline method~\cite{stachowicz2021ohmp}, while this baseline has similar runtime as ours, this baseline gets stuck in worse local minima for almost all instances when the dynamics is nonlinear, while our \abbrAl and the brute force baseline always find a better local minima with up to 7\% cheaper costs.

% our \abbrAl has similar runtime while finding up to 7\% cheaper solutions.

% Results show that, our approach always finds the same optimal solution as a naive brute force baseline method, while running up to 40 times faster.
% In comparison to the 


\begin{comment}
----------------------------------------

Our approach builds on the observation that the Riccati difference equation can be viewed as a Linear Fractional Transformation (LFT). By reformulating the backward pass in information form, we construct a "propagator" that allows us to compose the inverse value functions incrementally. This structure decouples the backward recursion from the specific terminal time $T$, allowing us to query the optimal cost for \textit{any} candidate horizon using the same pre-computed propagator sequence. 
Furthermore, to apply this logic to iLQR, we introduce an augmented state space formulation that absorbs the time-varying affine linearization terms. This unifies the treatment of linear and nonlinear problems, allowing the propagator to compute the \textit{exact} LQR cost for all horizons in a single $\mathcal{O}(N)$ pass.
 
The main contributions of this paper are:
\begin{enumerate}
    \item \textbf{Propagator-based Horizon Selection:} We develop an LFT-based solver that enables the reuse of backward pass computations, reducing the complexity of horizon selection from $\mathcal{O}(N^2n^3)$ to $\mathcal{O}(Nn^3)$.
    \item \textbf{Augmented State Formulation:} We propose a state augmentation technique that embeds affine linearization terms into a homogeneous coordinate system, extending the efficient propagator method to general nonlinear iLQR problems.
    \item \textbf{Performance and Robustness:} We validate our algorithm on four benchmark systems, including a 12-DOF Quadrotor. Experimental results show that our method achieves speedups of up to $43\times$ compared to brute-force search while guaranteeing global optimality with respect to the linearized model. 
\end{enumerate}



------------------------------------------


Existing approaches to this problem generally fall into two categories: continuous-time relaxations and discrete search. 
Relaxation methods treat the final time as a continuous decision variable, typically by scaling the system dynamics. However, this often introduces non-convexity into the optimization landscape, leading to poor convergence.
Discrete search methods, on the other hand, treat $T$ as an integer parameter. A naive "brute-force" strategy involves evaluating every candidate horizon $T \in [T_{min}, T_{max}]$. 



------------------

Time-optimal trajectory planning—generating motions that complete a task in the minimum possible time—is a fundamental requirement for agile robotic systems. From autonomous drone racing to emergency collision avoidance in self-driving cars, the ability to jointly optimize the control sequence and the total maneuver duration $T$ is critical for pushing physical limits. 
While Differential Dynamic Programming (DDP) and its variant, the iterative Linear Quadratic Regulator (iLQR), have become standard tools for high-dimensional trajectory optimization, they typically assume a fixed planning horizon. Extending these methods to time-optimal control introduces a discrete-continuous optimization challenge: the solver must determine the optimal integer horizon $T^*$ alongside the continuous control inputs.

Existing approaches to this problem generally fall into two categories: continuous-time relaxations and discrete search. 
Relaxation methods treat the final time as a continuous decision variable, typically by scaling the system dynamics. However, this often introduces non-convexity into the optimization landscape, leading to poor convergence.
Discrete search methods, on the other hand, treat $T$ as an integer parameter. A naive "brute-force" strategy involves evaluating every candidate horizon $T \in [T_{min}, T_{max}]$. 
\textbf{A fundamental bottleneck in this approach lies in the structure of the standard Riccati recursion.} In the LQR backward pass, the Value Function $V_k$ is computed recursively starting from a terminal cost anchored at the final time step $T$ (i.e., $P_T = Q_T$). Consequently, changing the horizon from $T$ to $T+1$ shifts the boundary condition, invalidating the entire sequence of previously computed Cost-to-Go matrices (P). This structural dependency prevents the reuse of historical computations across different horizons, forcing the solver to restart the backward pass from scratch for each candidate $T$, resulting in a prohibitive $\mathcal{O}(N^2)$ complexity.

To mitigate this computational burden, recent works such as the "One-Pass" method [1] have attempted to estimate costs for neighboring horizons by reusing the value function from a single nominal backward pass. While efficient for Linear Time-Invariant (LTI) systems where the dynamics do not shift with time, this approach fails for general trajectory optimization. In nonlinear systems, the local linearization ($A_k, B_k$) is time-varying; thus, reusing a fixed value function for different horizons introduces severe approximation errors, often leading to suboptimal horizon selection.

In this work, we propose a method that enables \textbf{exact computational reuse} for time-varying systems and extends it to the iLQR framework. 
Our approach builds on the observation that the Riccati difference equation can be viewed as a Linear Fractional Transformation (LFT). By reformulating the backward pass in information form, we construct a "propagator" that allows us to compose the inverse value functions incrementally. This structure decouples the backward recursion from the specific terminal time $T$, allowing us to query the optimal cost for \textit{any} candidate horizon using the same pre-computed propagator sequence. 
Furthermore, to apply this logic to iLQR, we introduce an augmented state space formulation that absorbs the time-varying affine linearization terms. This unifies the treatment of linear and nonlinear problems, allowing the propagator to compute the \textit{exact} LQR cost for all horizons in a single $\mathcal{O}(N)$ pass.
 
The main contributions of this paper are:
\begin{enumerate}
    \item \textbf{Propagator-based Horizon Selection:} We develop an LFT-based solver that enables the reuse of backward pass computations, reducing the complexity of horizon selection from $\mathcal{O}(N^2n^3)$ to $\mathcal{O}(Nn^3)$.
    \item \textbf{Augmented State Formulation:} We propose a state augmentation technique that embeds affine linearization terms into a homogeneous coordinate system, extending the efficient propagator method to general nonlinear iLQR problems.
    \item \textbf{Performance and Robustness:} We validate our algorithm on four benchmark systems, including a 12-DOF Quadrotor. Experimental results show that our method achieves speedups of up to $43\times$ compared to brute-force search while guaranteeing global optimality with respect to the linearized model. 
\end{enumerate}

\end{comment}

% \section{Related Work}\label{milp:sec:related}
% 
Horizon/time-optimal planning naturally arises in various robotic tasks.
For quadrotors, shorter horizon enables aggressive maneuvers and racing~\cite{mellinger2011minimumsnap,foehn2020alphapilot}.
For legged locomotion, optimization-based planners frequently re-solve constrained OC problems online to generate dynamically feasible motions, where the horizon length and replanning rate strongly affect robustness and responsiveness~\cite{kuindersma2016atlas}.
For information gathering, the horizon governs how far the robot plans ahead when trading immediate motion against long-term information gain~\cite{hollinger2013rig,time-otpimal-ergodic-search}.
Among these tasks, horizon/time choice is often embedded into the model predictive control loop, and fast horizon-optimal planning can be crucial for high-rate replanning~\cite{tassa2008rhdp,tassa2012synthesis}.

Horizon-optimal planning can be formulated as a nonlinear program (NLP) with the horizon included as a continuous decision variable~\cite{wang2022geometrically,dong2025time}.
% In practice, a common approach is direct transcription: discretize the dynamics and solve the resulting nonlinear program~\cite{betts1998survey,rao2009survey}.
The time-scaling approach fixes the horizon length $N$ and optimizes the length of time steps~\cite{rao2009survey,foehn2021time}.
These NLPs can incorporate various constraints, such as obstacle avoidance, and can be solved by large-scale interior-point methods such as IPOPT~\cite{wachter2002ipopt}.
However, the resulting nonconvex NLPs can be sensitive to initialization and discretization, and is often computationally heavy, which requires domain-specific fine-tuning to achieve high-frequency ``real-time'' computation.

Another strategy to address horizon-optimal planning is to extend well-known OC algorithms such as LQR and DDP to their horizon-optimal counterparts.
Early work analyzes both continuous-time and discrete-time horizon-optimal LQR and derives the optimality conditions in theory~\cite{verriest1991lqmt,elalami1998dlqmt}.
A recent work~\cite{stachowicz2021ohmp} observes that horizon-optimal LQR, with stationary dynamics and costs, can be solved within a single backward pass by shifting horizons and reusing value functions, which leads to an efficient algorithm, and was extended with DDP for nonlinear systems.
Other research~\cite{de2019free} combines LQR with bi-level optimization to shorten the horizon.

% However, for trajectory optimization in general, linearized dynamics are time-varying, which motivates this work.
% This motivates methods that remain valid for time-varying Riccati/DDP backward recursions and can query many candidate horizons without rerunning a full backward pass each time, which is the focus of this work.

% These works clarify the role of the terminal boundary condition in Riccati recursions, but they do not directly provide a general solution for time-varying or nonlinear system.

% For nonlinear robotics, DDP and iLQR compute locally optimal policies via a backward recursion on quadratic value-function approximations followed by a forward rollout~\cite{jacobson1970ddp,murray1984ddpnewton,li2004ilqr}, with MPC-style variants and constraint-handling extensions improving practicality~\cite{tassa2008rhdp,tassa2014clddp,xie2017ddp_constraints}.
% Several works further consider horizon selection while preserving this Riccati/DDP structure.


\begin{comment}
--------------

As discussed in the introduction, horizon/time-optimal optimal control (OC) requires solving for a trajectory while also deciding the maneuver duration.
Prior work typically addresses this coupling in two ways: (i) optimizing time directly together with controls in a single nonlinear program (NLP), or (ii) keeping a structured backward--forward solver (Riccati/DDP-style) and making horizon search efficient.


\subsection*{A. Free-final-time and direct time-scaling formulations}
Minimum-time and free-final-time OC are classically formulated in continuous time, where necessary conditions motivate optimizing the final time along with the control policy~\cite{pontryagin1962mtp,brysonho1975}.
In practice, a common approach is direct transcription: discretize the dynamics and solve the resulting nonlinear program~\cite{betts1998survey,rao2009survey}.
A standard free-final-time strategy is time-scaling: fix horizon length $N$ and optimize the total time (equivalently the time step $\Delta t$) together with the control sequence~\cite{betts1998survey,rao2009survey}.
These formulations are appealing because they can incorporate rich constraints (state limits, obstacle constraints, path constraints) without changing the overall NLP template, and can be solved by large-scale interior-point methods such as IPOPT~\cite{wachter2002ipopt}.
However, the resulting nonconvex NLPs can be sensitive to initialization and discretization, and often require substantial computation, which limits their use in high-frequency real-time settings.

\subsection*{B. Structured LQ minimum-time results and Riccati/DDP-style horizon search}
Minimum-time objectives under linear--quadratic (LQ) structure have been studied by explicitly trading horizon length against quadratic effort/terminal accuracy.
Verriest and Lewis analyze a continuous-time LQ minimum-time formulation and connect the optimal final time to Riccati Recursion~\cite{verriest1991lqmt}, while El Alami \emph{et al.} study a discrete-time version with a terminal constraint and an explicit time penalty~\cite{elalami1998dlqmt}.
These works clarify the role of the terminal boundary condition in Riccati recursions, but they do not directly provide a general solution for time-varying and nonlinear system.

For nonlinear robotics, DDP and iLQR compute locally optimal policies via a backward recursion on quadratic value-function approximations followed by a forward rollout~\cite{jacobson1970ddp,murray1984ddpnewton,li2004ilqr}, with MPC-style variants and constraint-handling extensions improving practicality~\cite{tassa2008rhdp,tassa2014clddp,xie2017ddp_constraints}.
Several works further consider horizon selection while preserving this Riccati/DDP structure.
Most closely related to our setting, Stachowicz and Theodorou propose an optimal-horizon DDP/MPC method that reuses a single backward pass under stationary dynamics and costs via ``shift-horizon'' evaluation~\cite{stachowicz2021ohmp}.
In general trajectory optimization, however, local models and costs become time-varying after linearization and quadratization, making the value function inherently time-dependent and limiting such reuse.
This motivates methods that remain valid for time-varying Riccati/DDP backward recursions and can query many candidate horizons without rerunning a full backward pass each time, which is the focus of this work.

\subsection*{C. Robotics applications}
Horizon/time-optimal planning is central to robotics tasks that must reach a goal quickly while respecting dynamics, constraints, and limited actuation~\cite{pontryagin1962mtp,brysonho1975}.
In quadrotor flight, shorter motion times tighten thrust and tilt constraints and change the feasibility margin, which directly affects aggressive maneuvers and racing performance~\cite{mellinger2011minimumsnap,foehn2020alphapilot}.
Time allocation also matters in point-to-point navigation and MPC, where optimizing the traversal time while maintaining feasibility has been studied through time-parameterized trajectory representations~\cite{rosmann2015teb}.
For legged locomotion, optimization-based planners frequently re-solve constrained OC problems online to generate dynamically feasible motions, where the horizon length and replanning rate strongly affect robustness and responsiveness~\cite{kuindersma2016atlas}.
In multi-robot settings, collision avoidance and coordination can impose tight timing constraints and motivate fast replanning over short horizons~\cite{vandenberg2011orca}.
Finally, in robotic information gathering, the horizon governs how far the robot plans ahead when trading immediate motion against long-term information gain~\cite{hollinger2013rig}.

Across these applications, horizon/time choice is often embedded into an MPC loop, so reducing per-iteration cost and enabling computational reuse can be crucial for high-rate replanning~\cite{tassa2008rhdp,tassa2012synthesis}.

\end{comment}

\section{Problem Description}\label{milp:sec:problem}
\input{problem}

\section{Preliminaries}\label{milp:sec:preli}
\input{preli}

\section{Optimal Horizon Time-Varying LQR}\label{milp:sec:method}
% \subsection{Method 1: Propagator-based Time-Optimal LQR via Linear Fractional Transformations}

% We present a novel approach for time-optimal control of linear time-varying (LTV) systems that achieves $\mathcal{O}(Nn^3)$ complexity—matching a single Riccati backward pass—while evaluating costs for all possible arrival times.

% \subsubsection{Key Insight and Motivation}






% \subsection{Our Approach for solving time optimal TVLQR problems}\label{sec:method1}
% To address the loss of reusability in time-varying systems, we shift from \emph{reusing values} to \emph{reusing mappings}.
% In the time-invariant case, different horizons reuse the same Riccati update and only change which $P_{N-t}$ is read in $J_t$ (Fig.~\ref{fig:reuse_and_prop}(a)).
% In the time-varying case, $g_k$ changes with $k$, so the \emph{values} $\{P_k\}$ are not reusable across horizons.

To address this challenge, our key idea (Fig.~\ref{fig:alg_flow}) is to rewrite the map $g_k$ as a new linear fractional transformation (LFT) form $\tilde{g}_{0:k}$ (which is explained later), and some of the matrices that help compute $\tilde{g}_{0:k}$ can be reused.
As a result, these matrices only need to be computed once for all possible horizons $k=1,2,\cdots,N$, as opposed to be repetitively computed for each possible horizon, which thus saves computational effort.

\subsection{Linear Fractional Transformation Form}
We first define necessary notations, and then rewrite the Riccati recursion into a new form of Linear Fractional Transformation (LFT) based on the inverse of cost-to-go matrices $P_k$.
We prove that this LFT form is equivalent to the original Riccati recursion in Theorem~\ref{thm:single_step_lft}.
Based on this result, we further derive a LFT form for the time-varying case of LQR in Theorem~\ref{thm:composed_lft}, which then leads to an efficient algorithm \abbrAlg for the time-varying HO-LQR problem as explained in the next subsection.

Let $\tilde P_k := P_k^{-1}$ denote the inverse of the cost-to-go matrix $P_k$.
Let $\tilde g_k$ denote the map from $\tilde P_{k+1}$ to $\tilde P_{k}$, i.e.,  $\tilde{P}_k = \tilde{g}_k(\tilde{P}_{k+1})$.
Let notation $\tilde g_{0:k}=\tilde g_0\circ\cdots\circ\tilde g_k$ denote a \textit{composed map} that composes the maps $\tilde g_0,\tilde g_1,\cdots,\tilde g_k$ sequentially, i.e., $\tilde{P}_0 = \tilde{g}_{0:k}(\tilde{P}_{k+1})$

\begin{theorem}[LFT form for Riccati Recursion] \label{thm:single_step_lft}
The Riccati recursion is equivalent to a Linear Fractional Transformation (LFT) on the inverse of $P_k$:
\begin{equation}\label{eq:single_step_lft}
    \tilde{g}_k(\tilde{P}) \triangleq E_k - F_k(\tilde{P} + G_k)^{-1}F_k^\top,
\end{equation}
such that $\tilde{P}_k = \tilde{g}_k(\tilde{P}_{k+1})$. where,
\begin{equation}
\begin{aligned}
    E_k = Q_k^{-1}, \;
    F_k = Q_k^{-1}A_k^\top, \;
    G_k = A_k Q_k^{-1}A_k^\top + B_k R_k^{-1}B_k^\top.
\end{aligned}
\end{equation}
\end{theorem}

\begin{proof}
We will use the Woodbury matrix identity: for invertible matrices $A$ and $C$, the following identity holds.
\begin{equation} \label{eq:woodbury_identity}
    (A + UCV)^{-1} = A^{-1} - A^{-1}U(C^{-1} + VA^{-1}U)^{-1}VA^{-1},
\end{equation}
where $A$, $C$, $U$ and $V$ are matrices of sizes $n_1\times n_1$, $n_2\times n_2$, $n_1\times n_2$, and $n_2\times n_1$ respectively.
% matrices: A is n×n, C is k×k, U is n×k, and V is k×n.

Our proof consists of two steps.
First, we rewrite the Riccati equation as:
\begin{align}
P_k &= Q_k + A_k^\top {\mathcal{K}} A_k
\end{align}
where, $\mathcal{K} = {\left[ P_{k+1} - P_{k+1} B_k (R_k + B_k^\top P_{k+1} B_k)^{-1} B_k^\top P_{k+1} \right]}$.
We recognize the term $\mathcal{K}$ as the right-hand side of the Woodbury identity.
By setting $A^{-1} = P_{k+1}$, $U = B_k$, $V = B_k^\top$, and $C^{-1} = R_k$, the identity \eqref{eq:woodbury_identity} implies:
\[
    \mathcal{K} = (P_{k+1}^{-1} + B_k R_k^{-1} B_k^\top)^{-1}.
\]
Substituting $\tilde{P}_{k+1} = P_{k+1}^{-1}$ and defining $S_k \triangleq B_k R_k^{-1} B_k^\top$, the Riccati equation becomes:
\begin{equation} \label{eq:intermediate_pk}
    P_k = Q_k + A_k^\top (\tilde{P}_{k+1} + S_k)^{-1} A_k.
\end{equation}
Next, we invert Eq.~\eqref{eq:intermediate_pk} to obtain the LFT form:
\begin{align}
    \tilde{P}_k = P_k^{-1} = \left( Q_k + A_k^\top (\tilde{P}_{k+1} + S_k)^{-1} A_k \right)^{-1}.
\end{align}
We apply the Woodbury identity \eqref{eq:woodbury_identity} again by setting $A = Q_k$, $U = A_k^\top$, $V = A_k$, and $C = (\tilde{P}_{k+1} + S_k)^{-1}$, which yields:
\begin{align}
    \tilde{P}_k &= Q_k^{-1} - Q_k^{-1} A_k^\top \Bigl( (\tilde{P}_{k+1} + S_k) + A_k Q_k^{-1} A_k^\top \Bigr)^{-1} A_k Q_k^{-1}.
\end{align}
Define $E_k = Q_k^{-1}$, $F_k = Q_k^{-1}A_k^\top$, $G_k = A_k Q_k^{-1}A_k^\top + B_k R_k^{-1}B_k^\top$, we obtain the desired LFT form.
\end{proof}

\begin{figure*}[t]
    \centering
    \includegraphics[width=0.8\textwidth]{source/figures/alg_flow.png}
    \vspace{-4mm}
    \caption{Illustration of our \abbrAlg Algorithm.
    {Green Blocks:} The system matrices are used to compute matrices $(E_k, F_k, G_k)$ and $(\overline{E}_k, \overline{F}_k, \overline{G}_k)$.
    {Top Row (Blue):} The maps $\tilde{g}_k(\cdot)$ forms the regular backward pass that maps $\tilde P_k$ to $\tilde P_{k-1}$.
    {Bottom Row (Purple):} The matrices $(\overline{E}_k, \overline{F}_k, \overline{G}_k)$ allow computing the composed maps $\tilde{g}_{0:k}(\cdot)$, which enables direct evaluation of the inverse of initial cost-to-go ($\tilde P_0$) for any horizon $k$.
    For example, to evaluate the cost for horizon $N-1$, let $\tilde{P}_{N-1}$ be the terminal cost-to-go matrix $\tilde{P_T}$. Then, the initial cost-to-go matrix is $\tilde{P_0}^{(N-1)} = \tilde{g}_{0:N-2}(\tilde{P}_{N-1})$. Then, the cost can be computed as $ J_{N-1} = \frac{1}{2} x_0^\top (\tilde{P}_0^{(N-1)})^{-1} x_0 + w \cdot (N-1) $.
    After evaluating all the time steps $J_k,k=1,2,\cdots,N$, we can find the minimal cost and select the corresponding time step as the optimal horizon.
    }
    \label{fig:alg_flow}
    \vspace{-3mm}
\end{figure*}

Theorem~\ref{thm:single_step_lft} allows us to rewrite the composite map into LFT form as well.
\begin{theorem}[LFT form of the composed maps]\label{thm:composed_lft}
There exist matrices \((\overline{E}_{k},\overline{F}_{k},\overline{G}_{k})\), $k=0,1,2,\cdots,N$ such that
\begin{equation}\label{eq:prefix_LFT}
\tilde{g}_{0:k}(\tilde{P}) \;=\; \overline E_{k} - \overline F_{k}\,(\tilde{P}+\overline  G_{k})^{-1} \overline F_{k}^\top,
\end{equation}
where, 
\begin{equation}\label{eq:prefix_recursion}
\begin{aligned}
W_k      &= (E_k + \overline G_{k-1})^{-1},\\
\overline E_{k}  &=\overline  E_{k-1} -\overline  F_{k-1} W_k\overline  F_{k-1}^\top,\\
\overline F_{k}  &= \overline F_{k-1} W_k F_k,\\
\overline G_{k}  &= G_k - F_k^\top W_k F_k,\\
\end{aligned}
\end{equation}
with \( \overline E_{0}=E_0,\; \overline F_{0}=F_0,\; \overline G_{0}=G_0\).
\end{theorem}

Note that $(A_k,B_k,Q_k,R_k)$ matrices are known as the input of the problem, and $(E_k,F_k,G_k,\overline{E}_k,\overline{F}_k,\overline{G}_k)$ are intermediate variables computed based on $(A_k,B_k,Q_k,R_k)$ and themselves recursively, and the composed map $\tilde{g}_k$ is computed based on $\overline{E}_k,\overline{F}_k,\overline{G}_k$ recursively.
We will explain this recursive computation later in Alg.~\ref{alg:ltv} with the help of Fig.~\ref{fig:alg_flow}.
We now prove the correctness of this theorem.

\begin{proof}
We prove Theorem~\ref{thm:composed_lft} by induction.
First, for the base case ($k=0$), it holds due to Theorem~\ref{thm:single_step_lft} and that $(\overline{E}_0, \overline{F}_0, \overline{G}_0) = (E_0, F_0, G_0)$.
Then, for the inductive step, assume the claim holds for $k-1$:
\begin{align}
\tilde{g}_{0:k-1}(\tilde{P}) = \overline{E}_{k-1} - \overline{F}_{k-1}(\tilde{P} + \overline{G}_{k-1})^{-1} \overline{F}_{k-1}^\top.\label{hop:eq:thm2_k-1}
\end{align}
By composition, $\tilde{g}_{0:k}(\tilde{P}) = \tilde{g}_{0:k-1} \big( \tilde{g}_k(\tilde{P}) \big)$. Substituting Eq.~\eqref{eq:single_step_lft} into the Eq.~\eqref{hop:eq:thm2_k-1} yields
\begin{equation}\label{eq:proof_sub}
\begin{aligned}
    \tilde{g}_{0:k}(\tilde{P}) &= \overline{E}_{k-1} - \overline{F}_{k-1} \Bigl( \bigl[ E_k - F_k(\tilde{P} + G_k)^{-1}F_k^\top \bigr] \\
    &\quad + \overline{G}_{k-1} \Bigr)^{-1} \overline{F}_{k-1}^\top.
\end{aligned}
\end{equation}
We simplify the inverse term between $\overline{F}_{k-1}$ and $\overline{F}_{k-1}^\top$. Let $W_k \triangleq (E_k + \overline{G}_{k-1})^{-1}$. The term to be inverted becomes:
\begin{equation}
    \mathcal{M} \triangleq \Bigl( W_k^{-1} - F_k(\tilde{P} + G_k)^{-1}F_k^\top \Bigr)^{-1}.
\end{equation}
Applying the Woodbury identity \eqref{eq:woodbury_identity} with $A=W_k^{-1}$:
\begin{equation} \label{eq:proof_woodbury}
    \mathcal{M} = W_k + W_k F_k \left( (\tilde{P} + G_k) - F_k^\top W_k F_k \right)^{-1} F_k^\top W_k.
\end{equation}
Substituting \eqref{eq:proof_woodbury} back into \eqref{eq:proof_sub} yields:
\begin{align*}
    \tilde{g}_{0:k}(\tilde{P}) &= \underbrace{ \overline{E}_{k-1} - \overline{F}_{k-1} W_k \overline{F}_{k-1}^\top }_{\overline{E}_k} \\
    &- \underbrace{ \overline{F}_{k-1} W_k F_k }_{\overline{F}_k}
    ( \tilde{P} + \underbrace{ G_k - F_k^\top W_k F_k }_{\overline{G}_k} )^{-1}
    \underbrace{ F_k^\top W_k \overline{F}_{k-1}^\top }_{\overline{F}_k^\top}.
\end{align*}
The resulting terms match the desired recursive form.
\end{proof}

\subsection{\abbrAlg Algorithm}


Based on Theorem~\ref{thm:composed_lft}, the complete \abbrAlg is summarized in Alg.~\ref{alg:ltv} and illustrated in Fig.~\ref{fig:alg_flow}.
\abbrAlg can be divided into two phases.

\subsubsection{Phase 1 Compute Composed Maps}
\abbrAlg performs a single forward sweep to build the parameters of $\tilde g_k$.
First, for each step $k$, we use system matrices $\{Q_i,R_i,A_i,B_i\}_{i=0}^N$ to compute the parameters $(E_k, F_k, G_k)$ (highlighted in green).
As shown in the Top Row (highlighted in blue)  of Fig.~\ref{fig:alg_flow}, these parameters fully define the single-step mapping $\tilde{g}_k$, which represents the LFT form of the Riccati equation (Theorem~\ref{thm:single_step_lft}).
Then, instead of executing the backward recursion immediately, we use the recursive formula in Theorem~\ref{thm:composed_lft} to accumulate these parameters forward, which helps compute the matrices $(\overline{E}_k, \overline{F}_k, \overline{G}_k)$ (highlighted in green) that will be used to compute the composed map $\tilde{g}_{0:k}$ from time $0$ to $k$.

\subsubsection{Phase 2 Fast Horizon Query}
With the composed maps $\tilde{g}_{0:k}$ stored, evaluating the cost $J_k$ for any horizon $k=1,2,\cdots,N$ becomes a straightforward evaluation of  $J_k$. 
We apply the stored map at index $t-1$ to the terminal condition $\tilde{P}_t = \tilde{P}_T$. Specifically, the initial inverse cost-to-go $\tilde{P}_0^{(t)}$ is:
\begin{equation}\label{eq:query_formula}
    \tilde{P}_0^{(t)} = \tilde{g}_{0:t-1}(\tilde{P}_T).
\end{equation}
Once $\tilde{P}_0^{(t)}$ is obtained, the total cost is given by
\begin{equation}\label{eq:J_t}
    J_t = \frac{1}{2} x_0^\top 
(\tilde{P}_0^{(t)})^{-1} x_0 + w \cdot t
\end{equation}
This effectively avoids the need to solve the chain of $\tilde{g}$ functions one by one.

With all $J_k$ computed, \abbrAlg can find the $k^*$ such that $J_{k^*}$ reaches the minimum, and this $k^*$ is the optimal horizon to the problem.
The controls can be obtained in the same way as in regular LQR:
\begin{align}
    u_k &= - \underbrace{ R_k^{-1} B_k^\top (\tilde{P}_{k+1} + B_k R_k^{-1} B_k^\top)^{-1} A_k }_{K_k} x_k, \label{eq:optimal_control}
\end{align}

\begin{algorithm}[t]
\SetAlgoLined
\DontPrintSemicolon
\small
\caption{\abbrAlg}
\label{alg:ltv}

\KwIn{System matrices $\{A_k, B_k, Q_k, R_k\}$, Initial state $x_0$, Terminal cost-to-go $\tilde{P}_T$ matrix, Time penalty $w$, Max horizon $N$}
\KwOut{Optimal costs $\{J_t\}_{t=1}^N$ for all arrival times}

\BlankLine
\tcp{Phase 1: Compute Composed Maps}
% \tcp{Step 1a: Compute single-step LFT matrices (Middle Row)}
\For{$k \gets 0$ \KwTo $N-1$}{
    \;$E_k \gets Q_k^{-1}$\\
    $F_k \gets Q_k^{-1} A_k^\top$\\
    $G_k \gets A_k Q_k^{-1} A_k^\top + B_k R_k^{-1} B_k^\top$\;
}

% \BlankLine
% \tcp{Step 1b: Accumulate prefix recursion (Bottom Row)}
Initialize: $\overline{E}_0 \gets E_0$, $\overline{F}_0 \gets F_0$, $\overline{G}_0 \gets G_0$\;

\For{$k \gets 1$ \KwTo $N-1$}{
    \;$W_k \gets (E_k + \overline{G}_{k-1})^{-1}$\\
    $\overline{E}_k \gets \overline{E}_{k-1} - \overline{F}_{k-1} W_k \overline{F}_{k-1}^\top$\\
    $\overline{F}_k \gets \overline{F}_{k-1} W_k F_k$\;
    $\overline{G}_k \gets G_k - F_k^\top W_k F_k$\;
}

% \BlankLine
\tcp{{Phase 2: Fast Horizon Queries}}
\For{$t \gets 1$ \KwTo $N$}{
    % \tcp{Retrieve stored propagator at $t-1$}
    \;$\tilde{P}_0 \gets \overline{E}_{t-1} - \overline{F}_{t-1} (\tilde{P}_T + \overline{G}_{t-1})^{-1} \overline{F}_{t-1}^\top$\\
    $P_0 \gets \tilde{P}_0^{-1}$\\
    $J_t \gets \frac{1}{2} x_0^\top P_0 x_0 + t \cdot w$\;
}

\Return{$\{J_t\}_{t=1}^N$,  $\arg\min_{t}J_t$}\;
\end{algorithm}

\subsection{Complexity Analysis}
Recall that $N$ is the number of all possible horizons to be chosen from, and $n$ is the dimensionality of the system's state $x$.
As aforementioned, a naive approach for the Horizon-Optimal Time-Varying LQR problem requires solving the Riccati equation repeatedly for every horizon $k=1,\dots,N$, which has a runtime complexity of $\mathcal{O}(N^2 n^3)$.

In contrast, our \abbrAlg only runs the forward recursion once  to compute the composed maps, and then performs simple queries based on these maps.
The runtime complexity for computing these composed map is $\mathcal{O}(N n^3)$ and the runtime complexity for the queries is $O(N n^3)$ as well.
The total runtime complexity is $\mathcal{O}(N n^3)$.
We summarize this result with the following theorem.

\begin{theorem}
With $N$ being the number of all possible horizons to be chosen from, and $n$ being the dimensionality of the system's state $x$, the proposed \abbrAlg algorithm has a runtime complexity of $\mathcal{O}(N n^3)$.
\end{theorem}

In other words, compared to the naive approach, \abbrAlg reduces the complexity from quadratic to linear with respect to the number of horizons $N$, which is of the same runtime complexity of the Riccati recursion for the fixed-horizon LQR.


% \begin{remark}
% Standard LQR solves a fixed-horizon problem via one backward Riccati recursion and yields the quadratic value function and linear feedback law~\cite{andersonmoore1990lqr}.
% A naive horizon scan repeats this recursion for each candidate horizon; ``shift-horizon'' reuse avoids repetition only under stationary dynamics and costs~\cite{stachowicz2021ohmp}.
% HOP-LQR preserves the same LQR structure, but enables exact reuse for time-varying models by rewriting the Riccati recursion as an LFT and composing $\tilde g_{0:k}$ to query all horizons in a single $\mathcal{O}(N)$ sweep.
% \end{remark}





\section{Optimal Horizon iLQR and DDP}\label{milp:sec:method2}

\subsection{Augmented Dynamics and Cost}
\label{sec:augmented_dynamics}

Standard DDP/iLQR~\cite{li2004ilqr} solves nonlinear OC by iteratively approximating them as linear-quadratic sub-problems.
Linearization introduces affine terms, which break the standard pure-quadratic LQR form ($x^\top P x$) required by our LFT-based method as aforementioned.
To address this, we perform a transformation including (1) approximation of the dynamics and cost function, (2) quadratization via completing the square, and (3) state augmentation.

\subsubsection{Approximation}
Given a nominal trajectory $(\bar{x}_k, \bar{u}_k)$, we define the deviations $\delta x_k := x_k - \bar{x}_k$ and $\delta u_k := u_k - \bar{u}_k$. The linearized dynamics are:
\begin{equation}
    \delta x_{k+1} = A_k \delta x_k + B_k \delta u_k + a_k,
\end{equation}
where $A_k = \nabla_x f(\bar{x}_k, \bar{u}_k)$, $B_k = \nabla_u f(\bar{x}_k, \bar{u}_k)$, and $a_k = f(\bar{x}_k, \bar{u}_k) - \bar{x}_{k+1}$.

The stage cost is Taylor expanded to the second order:
\begin{equation}
\begin{aligned}
\ell_k &\approx \ell(\bar{x}_k, \bar{u}_k) + w + \ell_{x,k}^\top \delta x_k + \ell_{u,k}^\top \delta u_k \\
&\quad + \tfrac{1}{2} \delta x_k^\top \ell_{xx,k} \delta x_k + \delta x_k^\top \ell_{xu,k} \delta u_k \\
&\quad + \tfrac{1}{2} \delta u_k^\top \ell_{uu,k} \delta u_k,
\end{aligned}
\end{equation}
which are partial derivatives evaluated at $(\bar{x}_k, \bar{u}_k)$, and the time penalty $w$ is included as a constant term.

\subsubsection{Quadratization}
The cross-term $\delta x_k^\top \ell_{xu,k} \delta u_k$ prevents direct application of standard LQR methods. To eliminate it, we complete the square for the control terms.
We first group all terms dependent on $\delta u_k$:
\[
\tfrac{1}{2} \delta u_k^\top \ell_{uu,k} \delta u_k + \delta u_k^\top (\ell_{ux,k} \delta x_k + \ell_{u,k}).
\]
Recalling the identity $\frac{1}{2}v^\top R v = \frac{1}{2}(u + R^{-1}b)^\top R (u + R^{-1}b) = \frac{1}{2}u^\top R u + u^\top b + \text{const.}$, we can match terms to define the new control variable $v_k$:
\begin{equation}\label{eq:new_control}
    v_k := \delta u_k + \ell_{uu,k}^{-1}(\ell_{ux,k} \delta x_k + \ell_{u,k}).
\end{equation}
Substituting $v_k$ back into the cost function allows us to express the stage cost in a decoupled form. The expansion is:
\begin{equation}
\begin{aligned}
\ell_k &\approx \tfrac{1}{2} \delta x_k^\top (\ell_{xx,k} - \ell_{xu,k} \ell_{uu,k}^{-1} \ell_{ux,k}) \delta x_k \\
&\quad + (\ell_{x,k} - \ell_{xu,k} \ell_{uu,k}^{-1} \ell_{u,k})^\top \delta x_k \\
&\quad + \tfrac{1}{2} v_k^\top \ell_{uu,k} v_k \\
&\quad + \left( \ell(\bar{x}_k, \bar{u}_k) + w - \tfrac{1}{2} \ell_{u,k}^\top \ell_{uu,k}^{-1} \ell_{u,k} \right).
\end{aligned}
\end{equation}

\subsubsection{Augmentation}
Even after using the new control variable, the dynamics and cost still contain state-dependent affine terms. We eliminate these by lifting the system into a homogeneous coordinate space via the augmented state
$z_k = [\delta x_k,\; 1]^T$.
This allows us to absorb all linear and constant terms into the quadratic form.

\paragraph{Augmented Dynamics}
Substituting the new control variable $v_k$ (Eq.~\ref{eq:new_control}) into the dynamics yields the explicit update equation:
\begin{equation}
\begin{aligned}
    \delta x_{k+1} &= \left( A_k - B_k \ell_{uu,k}^{-1} \ell_{ux,k} \right) \delta x_k + B_k v_k \\
    &\quad + \left( a_k - B_k \ell_{uu,k}^{-1} \ell_{u,k} \right).
\end{aligned}
\end{equation}
We then rewrite it based on $z_k$  and get the new augmented system dynamics:
\begin{equation}
    % \boxed{
    z_{k+1} = A_k^{\text{aug}} z_k + B_k^{\text{aug}} v_k,
    % }
\end{equation}
\begin{equation}\label{eq:aug_dynamics}
    A_k^{\text{aug}} = \begin{bmatrix}
    A_k - B_k \ell_{uu,k}^{-1} \ell_{ux,k} & a_k - B_k \ell_{uu,k}^{-1} \ell_{u,k} \\
    0 & 1
    \end{bmatrix},
\end{equation}
\begin{equation}
    B_k^{\text{aug}} = \begin{bmatrix}
    B_k \\ 0
    \end{bmatrix}.
\end{equation}

\paragraph{Augmented Cost}
Similarly, we define notations for the modified cost terms:
\[
\tilde{Q}_k = \ell_{xx,k} - \ell_{xu,k} \ell_{uu,k}^{-1} \ell_{ux,k}, \quad
\tilde{q}_k = \ell_{x,k} - \ell_{xu,k} \ell_{uu,k}^{-1} \ell_{u,k}.
\]
Then the augmented cost matrix becomes:
\begin{equation}\label{eq:aug_cost_matrices}
Q_k^{\text{aug}} = \begin{bmatrix}
\tilde{Q}_k & \tilde{q}_k \\
\tilde{q}_k^\top & 2\left(\ell(\bar{x}_k, \bar{u}_k) + w - \tfrac{1}{2} \ell_{u,k}^\top \ell_{uu,k}^{-1} \ell_{u,k}\right)
\end{bmatrix}.
\end{equation}
\[
R_k = \ell_{uu,k}.
\]

The stage cost in augmented form is quadratic:
\begin{equation}\label{eq:aug_stage_cost}
\ell_k \approx \tfrac{1}{2} z_k^\top Q_k^{\text{aug}} z_k + \tfrac{1}{2} v_k^\top R_k v_k
\end{equation}

Similarly, the terminal cost matrix is:
\begin{equation}\label{eq:aug_terminal_cost}
Q_N^{\text{aug}} = \begin{bmatrix}
\phi_{xx,N} & \phi_{x,N} \\
\phi_{x,N}^\top & 2\phi(\bar{x}_N)
\end{bmatrix}.
\end{equation}

% ------------------------------------------------------------------





\begin{algorithm}[tb]
\SetAlgoLined
\DontPrintSemicolon
\caption{\abbrAlgg Algorithm}
\label{alg:time-optimal-ilqr}
\small
\KwIn{Dynamics $f$, costs $\ell, \phi$, initial state $x_0$, initial controls $U$, horizon bounds $[T_{\min}, T_{\max}]$, time penalty $w$, $N = T_{\max}$ }
\KwOut{Optimal trajectory and controls}

\Repeat{convergence}{
    \tcp{Step 1: Linearization and augmentation}
    Rollout $X$ using $f$ and $U$\;
    \For{$k = 0$ \KwTo $N-1$}{
        Compute $A_k, B_k$ at $(\bar{x}_k, \bar{u}_k)$\\
        Compute derivatives: $\ell_{x,k}, \ell_{u,k}, \ell_{xx,k}, \ell_{ux,k}, \ell_{uu,k}$\\
        Compute $A_k^{\text{aug}}, B_k^{\text{aug}}, Q_k^{\text{aug}}, R_k$ via Eqs.~\eqref{eq:aug_dynamics}-\eqref{eq:aug_cost_matrices}\;
    }
    Build $Q_T^{\text{aug}}$ from $\phi$ at $\bar{x}_T$\;
    
    \BlankLine
    \tcp{Step 2: Horizon selection}
    $J \gets$ \abbrAlg$(A^{\text{aug}}, B^{\text{aug}}, Q^{\text{aug}}, R, z_0, Q_T^{\text{aug}}, N)$\\
    $T^* \gets \arg\min_{t \in [T_{\min}, T_{\max}]} J[t]$\;
    
    \BlankLine
    \tcp{Step 3: Backward pass on $[0, T^*-1]$}
    % Initialize: 
    
    $V_{xx}[T^*] \gets \phi_{xx}$, $V_x[T^*] \gets \phi_x$, $V_0[T^*] \gets \phi(\bar{x}_{T^*})$\\
    \For{$k = T^*-1, T^*-2, \cdots,0$}{
        \tcp{Compute derivatives (DDP terms in parentheses, iLQR ignores them)}
        $Q_x  \gets \ell_{x,k} + A_k^\top V_{x,k+1}\; \bigl(+\, A_k^\top V_{xx,k+1}\, a_k \bigr)$\\
        $Q_u  \gets \ell_{u,k} + B_k^\top V_{x,k+1}\; \bigl(+\, B_k^\top V_{xx,k+1}\, a_k \bigr)$\\        
        $Q_{xx} \gets \ell_{xx,k} + A_k^\top V_{xx,k+1} A_k\;
        \bigl(+\, \sum_{i=1}^{n_x} V_{x,k+1}^{(i)}\, f_{xx,k}^{(i)} \bigr)$\\    
        $Q_{ux} \gets \ell_{ux,k} + B_k^\top V_{xx,k+1} A_k\;
        \bigl(+\, \sum_{i=1}^{n_x} V_{x,k+1}^{(i)}\, f_{ux,k}^{(i)} \bigr)$\\
        $Q_{uu} \gets \ell_{uu,k} + B_k^\top V_{xx,k+1} B_k\;
        \bigl(+\, \sum_{i=1}^{n_x} V_{x,k+1}^{(i)}\, f_{uu,k}^{(i)} \bigr)$\\
        $Q_{xu} \gets Q_{ux}^\top$\\
        $Q_{uu} \gets Q_{uu} + \lambda I$ \tcp*{LM regularization}

        
        \tcp{Compute gains}
        $\kappa_k \gets -Q_{uu}^{-1} Q_u$, $K_k \gets -Q_{uu}^{-1} Q_{ux}$
        
        \tcp{Update value function}
        $V_{xx,k} \gets Q_{xx} - Q_{ux}^\top Q_{uu}^{-1} Q_{ux}$\\
        $V_{x,k} \gets Q_x - Q_{ux}^\top Q_{uu}^{-1} Q_u$\\
    $V_{0,k} \gets V_{0,k+1} + \ell(\bar{x}_k,\bar{u}_k) + w
          + \kappa_k^\top Q_u + \tfrac{1}{2}\kappa_k^\top Q_{uu}\kappa_k$

    }
    
    \BlankLine
    \tcp{Step 4: Forward rollout with line search}
    \For{$\alpha \in I_{\alpha}$}{
        $x_{\text{new}}[0] \gets x_0$\\
        \For{$k = 0$ \KwTo $T^*-1$}{
            $\delta x \gets x_{\text{new}}[k] - \bar{x}_k$\\
            $\delta u \gets K_k \cdot \delta x + \alpha \cdot \kappa_k$\\
            $u_{\text{new}}[k] \gets \bar{u}_k + \delta u$\\
            $x_{\text{new}}[k+1] \gets f(x_{\text{new}}[k], u_{\text{new}}[k])$\;
        }
        \If{TrueCost$(x_{\text{new}}, u_{\text{new}}, T^*) <$ TrueCost$(\bar{x}, \bar{u}, T^*)$}{
            Accept trajectory; \textbf{break}\;
        }
    }
    \If{not accepted}{$\lambda *= 10$ \tcp*{Increase LM regularization}}
    \Else{Update nominal $(X, U)$}
}
\end{algorithm}

\subsection{\abbrAlgg Algorithm}

We now present the complete \abbrAlgg Algorithm. With the augmented state space derived in Section~\ref{sec:augmented_dynamics}, the iterative optimization can be described in four steps (Alg.~\ref{alg:time-optimal-ilqr}).

\subsubsection{Step 1: Linearization and Augmentation}
The iteration begins by linearizing the nonlinear dynamics and quadratizing the cost function around the current nominal trajectory $(\bar{x}, \bar{u})$ as mentioned in Sec.~\ref{sec:augmented_dynamics}, which yields the augmented system matrices $A_k^{\text{aug}}, B_k^{\text{aug}}$ (Eq.~\ref{eq:aug_dynamics}) and the augmented cost matrices $Q_k^{\text{aug}}, R_k$ (Eq.~\ref{eq:aug_stage_cost}).

% This transformation yields a sequence of linear-quadratic parameters $\left\{ (A_k^{\text{aug}}, B_k^{\text{aug}}, Q_k^{\text{aug}}, R_k) \right\}_{k=0}^{N-1}$ for the entire horizon. 
% This sequence serves as the direct input for the \abbrAlg Algorithm in the next step.

\subsubsection{Step 2: Horizon Selection}
With the system $\left\{ (A_k^{\text{aug}}, B_k^{\text{aug}}, Q_k^{\text{aug}}, R_k) \right\}_{k=0}^{N-1}$ in a standard linear-quadratic form, we can now find an optimal horizon $T^*$ by calling the aforementioned \abbrAlg.
% solve the horizon selection sub-problem via \abbrAlg Algorithm.
This step identifies the optimal stopping time $T^*$ without requiring a full backward pass for each candidate horizon.

% Unlike standard iLQR which assumes a fixed $T$, we must identify the optimal time-horizon $T^*$ that minimizes the  cost.

% Using the \abbrAlg Algorithm (Method 1), we efficiently evaluate the optimal cost-to-go $J_t$ for every candidate horizon $t \in [T_{\min}, T_{\max}]$ simultaneously:
% \begin{equation}
% T^* = \arg\min_{t \in [T_{\min}, T_{\max}]} \left( \tfrac{1}{2} z_0^\top (\tilde{P}_0^{(t)})^{-1} z_0 \right).
% \end{equation}
% This step identifies the optimal stopping time $T^*$ without requiring a full backward pass for each candidate.

\subsubsection{Step 3: Truncated Backward Pass}
Once $T^*$ is known, we perform the \textit{truncated} backward pass, which is executed only on the interval $[0, T^*-1]$, rather than the maximum possible horizon $N$.
Specifically, we initialize the value function at $T^*$ using the terminal cost $\phi(\bar{x}_{T^*})$. Then, proceeding backward from $k = T^*-1$ to $0$, we compute the derivatives of $Q$-matrices and extract the feedback gains as in regular DDP:
\begin{equation}
    K_k = -Q_{uu,k}^{-1} Q_{ux,k}, \quad \kappa_k = -Q_{uu,k}^{-1} Q_{u,k}.
\end{equation}
This yields the modification on the current control $\delta u_k = K_k \delta x_k + \kappa_k$ for the selected horizon $T^*$.

\subsubsection{Step 4: Forward Rollout and Update}
Finally, we apply the computed gains to the original nonlinear system $x_{k+1} = f(x_k, u_k)$ to generate a new trajectory.
We perform a line search on the step size $\alpha \in I_{\alpha}$ to ensure convergence, where $I_{\alpha}$ is a user-specified set of possible step sizes.
To generate the new trajectory, we compute the new controls as follows.
\begin{equation}
u_k^{\text{new}} = \bar{u}_k + K_k(x_k^{\text{new}} - \bar{x}_k) + \alpha \kappa_k.
\end{equation}
Based on the computed new controls, a new state trajectory can be rolled out. 
This rollout is terminated at time step $T^*$.
If the total nonlinear cost of the trajectory decreases, the new trajectory is accepted as the nominal trajectory for the next iteration; otherwise, the regularization parameter is increased similarly to \cite{todorov2005ilqg}.

% \vspace{0.5em}


% \begin{remark}
% iLQR/DDP solves the fixed-horizon nonlinear OC by alternating linearization/quadratization, a backward pass, and a forward rollout~\cite{jacobson1970ddp,murray1984ddpnewton,li2004ilqr}.
% For horizon-optimal nonlinear OC, the literature~\cite{stachowicz2021ohmp} accelerates horizon selection by evaluating horizons approximately using a backward pass around a nominal horizon.
% This nominal horizon is computed by the one-pass approach that approximates the system as a time-invariant LQR.
% ???
% , which can be inaccurate when the local models change across horizons/iterations~\cite{stachowicz2021ohmp}.
% HOP-DDP instead performs exact horizon querying for the current LQ approximation via the augmented-state reduction and HOP-LQR, then runs the backward pass only on $[0,T^*-1]$.
% \end{remark}

\section{Experimental Results}\label{milp:sec:result}

\begin{figure*}[t]
    \centering
    \includegraphics[width=0.93\linewidth]{source/figures/exp1_all.png}
    \caption{Experimental results. 
    % \vspace{-5mm}
    (a) Comparison of runtime in log scale. Error bars indicate variability across trials. 
    % Both ours and Baseline-2 (OP) run faster than Baseline-1 (BF) and Baseline-3 (NLP). 
    (b) Comparison of speedup relative to Baseline-1 (BF) over the four systems. This figure shows the same results in (a) in a different way. 
    (c) Success rates of different algorithms. 
    % While all methods achieve high success rates on linear systems (Double Integrator and Segway), ours and Baseline-1 (BF) achieves higher success rates than Baseline-2 (OP) and Baseline-3 (NLP). 
    (d) Comparison of solution cost increases relative to Baseline-1 (BF). Our method finds solutions of the better quality than OP and NLP, while enjoying fast running speed.
    % Baseline-2 (OP) often finds more expensive solutions.
    }
    \label{fig:exp1_all_combined}
    \vspace{-3mm}
\end{figure*}


We compare our \abbrAlgg against several baselines on four systems including linear and nonlinear dynamics: Double Integrator, Segway Balance, Cartpole Swing-Up, and a 12-DOF Quadrotor with 25 cases each.
Here, Double Integrator and Segway (linearized about the equilibrium) have two linear dynamics, while Cartpole and Quadrotor have nonlinear dynamics.
We use three baseline methods for comparison.
\begin{enumerate}
    % \item \textbf{Ours (\abbrAlgg Algorithm):} The proposed method using the Augmented State and Horizon Selection via \abbrAlg Algorithm.
    \item {Baseline-1 (BruteForce, BF)} evaluates all horizons via backward Riccati recursions to select the horizon.
    \item {Baseline-2 (OnePass, OP) \cite{stachowicz2021ohmp}} uses time-invariant LQR to approximate nonlinear systems.
    \item {Baseline-3 (NLP)} includes the time horizon as a decision variable, uses time-scaled transcription, and solves the resulting NLP using IPOPT~\cite{wachter2002ipopt}.
\end{enumerate}


\subsection{Experiment 1: Overall Performance}

\paragraph{Runtime and Speedup}
Fig.~\ref{fig:exp1_all_combined}(a) reports the runtime on log scale, and Fig.~\ref{fig:exp1_all_combined}(b)
shows the corresponding speedup relative to Baseline-1 (BF).
For all systems, our method consistently runs faster than Baseline-1 (BF): \textbf{$\sim 9\times$} faster on Double Integrator, \textbf{$\sim 20\times$} faster on Segway Balance,
\textbf{$\sim 5\times$} faster on Quadrotor, and \textbf{$\sim 40\times$} faster on Cartpole.
Baseline-2 (OP) can be even faster on some tasks (e.g. Cartpole and Segway), but at the cost of worse solution quality as explained later.
Baseline-3 (NLP) usually has a similar runtime to Baseline-1 (BF), yet runs slower on some systems, which indicates the expensive computation of solving a large nonlinear program.

\paragraph{Success Rates}
Fig.~\ref{fig:exp1_all_combined}(c) reports success rates, which is the percentage of trials where the cost function converges and the system's terminal state is within a small error threshold from the goal state, i.e., $\|x_{T^*} - x_g\| \leq 0.5$.
All methods achieve high success rates on Double Integrator and Segway, which are linear systems.
% indicating that these tasks are well-conditioned under our initialization and cost setup.
For Quadrotor, all methods remain above $90\%$ success rates, suggesting that the task is feasible for a wide range of horizons and that local optimization is relatively stable once a reasonable rollout is found.
Cartpole here is the most challenging case, and our method and the Baseline-1 (BF) succeed for most
trials, while Baseline-2 (OP) exhibits much lower success rates, due to the mis-selection of horizon caused by the time-invariant LQR approximation of nonlinear systems.
Baseline-3 (NLP) has success rates higher than OP yet lower than Ours and Baseline-1 (BF), as the underlying NLP can get trapped in local minima.



\paragraph{Solution Quality}
Fig.~\ref{fig:exp1_all_combined}(d) reports the normalized cost increase ratio relative to the solution costs found by Baseline-1 (BF).
Here, Baseline-1 (BF), Baseline-2 (OP) and our methods all share the same objective function, while Baseline-3 (NLP) has slightly different objective functions due to the different formulation, and we thus remove it from the figure for clarity.
% computed under the same iLQR objective used by Ours and Baseline-2.
As shown in Fig.~\ref{fig:exp1_all_combined}(d), our method always finds almost the same solution costs as Baseline-1 (BF) for all systems, since the cost increases are 0\%.
It indicates that the fast horizon querying in our method does not compromise the final solution quality.
In contrast, Baseline-2 (OP) often finds more expensive solutions for those nonlinear systems, due to the approximation using time-invariant LQR.
% ng horizon evaluation with a single backward pass can miss the best horizon when local models vary strongly over time and across iterations.
% We do not include Baseline-3 (Direct-NLP, free $T$) in this cost comparison because it optimizes a different objective. Its discretized cost and constraints are not identical to the
% Riccati/DDP-style iLQR objective used by the other methods, so we omit it for a fair cost-optimality comparison.


\subsection{Experiment 2: Case Study on Quadrotor}
\begin{figure}[t]
    \centering
    \includegraphics[width=\linewidth]{source/figures/Quadrotor_Hover_cost_curve.png}
    % \vspace{-5mm}
    \caption{Cost values for various horizons for the hovering task of Quadrotor. Our method and Baseline-1 (BF) both finds an optimal horizon $T^*=32$ whose corresponding optimal costs are $J_{32}^{\text{ours}}\approx 484.79, J_{32}^{\text{BF}}\approx 484.80$, while Baseline-2 (OP) converges to a different local minimum with a longer horizon ($T\approx 74$) and $5.97\%$ higher cost ($J_{74}^{\text{OP}}\approx 513.75$).}
    \label{fig:exp2_quad_costcurve}
    % \vspace{-5mm}
\end{figure}

\begin{figure}[t]
    \centering
    \includegraphics[width=\linewidth]{source/figures/Quadrotor_Hover_timing.png}
    % \vspace{-5mm}
    \caption{Runtime breakdown for the hovering task of Quadrotor. Ours and Baseline-2 (OP) run faster than Baseline-1 (BF) as they bypass the expensive computation for horizon selection.}
    \label{fig:exp2_quad_timing}
    % \vspace{-5mm}
\end{figure}


We then look into a hovering task for Quadrotor.
We compare the solution costs and breakdown the runtime to explain why ours is as accurate as BF and as fast as OP.
% to illustrate how horizon selection affects both optimality and runtime.

\paragraph{Horizon optimality}
During the iterations of Alg.~\ref{alg:time-optimal-ilqr}, both ours and Baseline-1 (BF) need to iteratively select a horizon $T^*$ and solve.
Fig.~\ref{fig:exp2_quad_costcurve} shows the cost $J$ for all possible horizons during the iteration.
Baseline-2 (OP) has a different computational process and Fig.~\ref{fig:exp2_quad_costcurve} only shows the costs $J$ of horizons in its last iteration.

We observe that, our \abbrAlg and Baseline-1 (BF) have almost the same cost functions values $J$ for the entire horizon range, and select the same optimal horizon $T^*=32$ whose corresponding optimal costs are $J_{32}^{\text{ours}}\approx 484.79, J_{32}^{\text{BF}}\approx 484.80$.
In contrast, Baseline-2 (OP) converges to a local minimum and selects a longer horizon ($T\approx 74$) with higher cost ($J_{74}^{\text{OP}}\approx 513.75$), which is about {5.97\%} higher than the cost returned by Baseline-1 (BF).
% This case highlights that, on strongly nonlinear and coupled dynamics such as a 12-DOF quadrotor, accurate horizon evaluation is critical for avoiding suboptimal horizon choices.

\paragraph{Runtime breakdown}
We then look at the runtime breakdown of the algorithms.
Fig.~\ref{fig:exp2_quad_timing} explains why our algorithm runs faster.
% the computational savings come from.
Baseline-1 (BF) spends most of the runtime in selecting the horizon, since it must evaluate many candidate
horizons via repeated backward Riccati recursions.
Baseline-2 (OP) runs fast because it avoids the exhaustive horizon selection.
Similarly, our method can also bypass this expensive selection by using \abbrAlg for horizon selection. 
% It features a lightweight computation, shifting the runtime bottleneck to the shared
% Linearize step that all methods must perform.
% , but the cost curve in Fig.~\ref{fig:exp2_quad_costcurve} shows that this speed can come at the expense of selecting a non-optimal horizon.
% Overall, this case study demonstrates that \abbrAlgg Algorithm achieves near brute-force horizon optimality while substantially
% reducing the horizon-selection cost.


\section{Conclusion and Future Work}\label{milp:sec:conclude}



This paper develops a new approach \abbrAl for horizon-optimal OC problems with linear and nonlinear dynamics.
The key insight is to rewrite the Riccati recursion into a LFT form which allows computational reuse.
For time-varying HO-LQR problems, \abbrAlg reduces the runtime complexity from quadratic to linear with respect to the maximum possible horizon $N$.
For nonlinear system, the proposed \abbrAlgg runs fast while finding high-quality solutions.
We plan to include more sophisticated constraints such as collision avoidance into our \abbrAl in our future work.

% \section*{Acknowledgments}
% Acknowledgments omitted for anonymous review.

\bibliographystyle{plainnat}
\bibliography{references}

\end{document}